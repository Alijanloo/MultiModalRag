% Template for Persian sections to be included in main.tex
% 
% Note: If you want to cite something, add the reference in the main.tex bibliography section
% and then use \cite{reference-key} in this template file.

\section{مقدمه}
این بخش شامل مقدمه و توضیح کلی پروژه است. در اینجا می‌توانید اهداف، روش‌شناسی و ساختار کلی کار را توضیح دهید. مطالعات قبلی در این زمینه نشان می‌دهد که الگوریتم‌های زمان‌بندی کاربردهای گسترده‌ای دارند \cite{ref-1}.

\noindent
نکات مهم:
\begin{itemize}
    \item نکته اول
    \item نکته دوم
    \item نکته سوم
\end{itemize}

\noindent
ویژگی‌های اصلی این پروژه عبارتند از:
\begin{itemize}
    \item \textbf{ویژگی اول:} توضیح ویژگی اول.
    \item \textbf{ویژگی دوم:} توضیح ویژگی دوم.
    \item \textbf{ویژگی سوم:} توضیح ویژگی سوم.
\end{itemize}

\newpage

\section{روش‌شناسی}

\subsection{مدل‌سازی}
در این بخش، مدل ریاضی مسئله ارائه می‌شود.

\subsection*{متغیرها}
\begin{itemize}
    \item $ x $: متغیر اول
    \item $ y $: متغیر دوم
    \item $ z $: متغیر سوم
\end{itemize}

\subsection*{پارامترها}
\begin{itemize}
    \item $ a $: پارامتر اول
    \item $ b $: پارامتر دوم
    \item $ c $: پارامتر سوم
\end{itemize}

\subsection*{تابع هدف}
حداقل‌سازی تابع هدف:
\[
\text{Minimize} \quad f(x,y,z) = ax + by + cz
\]

\subsection*{محدودیت‌ها}
\begin{enumerate}
    \item محدودیت اول:
    \[
    x + y \leq 10
    \]
    \item محدودیت دوم:
    \[
    y + z \geq 5
    \]
    \item محدودیت سوم:
    \[
    x, y, z \geq 0
    \]
\end{enumerate}

\section{نتایج}
در این بخش نتایج حاصل از پیاده‌سازی و آزمایش‌ها ارائه می‌شود. نتایج به‌دست‌آمده با مطالعات پیشین مطابقت دارد \cite{ref-1}.

\begin{figure}[h]
    \centering
    \includegraphics[width=0.7\textwidth]{example_image.png}
    \caption{عنوان تصویر}
    \label{fig:example}
\end{figure}

\section{نتیجه‌گیری}
خلاصه‌ای از نتایج و دستاورد‌های پروژه در این بخش ارائه می‌شود.
